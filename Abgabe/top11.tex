\documentclass{scrartcl}
\usepackage{mathe-blatt}
\blatttop
\newcommand{\id}{\operatorname{id}}
\newcommand{\homo}{\cong}

\begin{document}
\setcounter{section}{11}
\setcounter{aufgabe}{1}
\begin{aufgabe}
\begin{enumerate}[(i)]
\item z.z. $ \R \ncong \R^n $\\
GA: $ \exists $ Homöomorphismus $ \Psi $.

Dann ergibt sich der Widerspruch da, $ \R\setminus\{0\} $ nicht zusammenhängend ist. ($ \R_-\cup \R_+ $ wäre eine Zerlegung in nichtleere offene Menge). Nun ist aber auf der anderen Seite $ \R^n\setminus\{\Phi(0)\} $ zusammenhängend. (sogar wegzusammenhängend)

\item \fixme[Hier fehlt was]
\end{enumerate}
\end{aufgabe}

\begin{aufgabe}
\begin{enumerate}[(i)]
\item 
\begin{enumerate}
\item $ X $ ist wegzusammenhängend und damit auch zusammenhängend:\\

\item Wir wollen zeigen, dass $ \pi_1(X)=\const. $:

Zunächst wählen wir $ x_0\in U\cap V $. Wir konstruieren uns $ 0=s_0<s_1<...<s_{2m}=1 $ mit $ x_n:=c(s_n) $, wobei $ x_{2n}\in U\cap V, x_{4n+1} \in U\setminus V, x_{4n+3}\in V\setminus U $, sowie 
\[
c([s_{4n}, s_{4n+2}])\subset U, c([s_{4n+2}, s_{4n+4}\subset V
\]
Verbinden wir die Endpunkte von $c([s_{4n}, s_{4n+2}])$ mit $ x_0 $, so folgt, dass diese homotop mit festen Endpunkten zu $ x_0 $ ist. Dies können wir für alle Teilstücke fortsetzen und damit gibt es eine Homotopie mit festen Endpunkten zur konstanten Abbildung.  Und damit ist die Fundamentalgruppe von $ X $ konstant.
\end{enumerate}
Damit ist $ X $ einfach zusammenhägend
\item Wir konstruieren eine Überdeckung von $ X $. Sei $ J=\{0,1\}^{n+1} $  $ S^n=\bigcup_{j\in J}$ mit einem $ \epsilon >0 $: 
\[
 U_j:= \{x=(x_1,...,x_{n+1}) \in S^n|j_i x_i > - \epsilon\}, \text{ mit } j \in J
\]
Dann ist $ U_j $ offensichtlich zusammenziehbar und damit einfach zusammenhängend.  Die Schnitte sind desweiteren wegzusammenhängend. $ S^n $ ist wegzusammenhägend. Damit sind die Voraussetzungen für den Satz erfüllt. und Damit ist $ S^n $ einfach zusammenhängend und damit ist $ \pi_1(S^n)=\{1\} $
\end{enumerate}
\end{aufgabe}

\begin{aufgabe}
\begin{enumerate}[(i)]
\item An dieser Stelle einige Beispiele für Kategorien:
\begin{itemize}
\item Die Menge der Vektorräume, mit der Menge von linearen Abbildungen zwischen zwei Vektorräumen, wobei das Einselement die Identität ist.
\item Die Menge der Vektorräume, mit der Menge von Matrizen, wobei das Einselement die Einheitsmatrix ist.
\item Die Klasse der Teilmengen von $ \R $, mit der Menge von glatten Abbildungen zwischen zwei Mengen, wobei das Einselement die Identität ist. 
\end{itemize}
\item Tatsächlich lassen sich Homotopien verknüpfen, diese ist assoziativ. Zudem ist das Einselement, die Idenität. (Homotopieäquivalenz ist reflexiv)
\item \fixme[Hier fehlt was]
\item \fixme[Hier fehlt was]
\end{enumerate}  
\end{aufgabe}
\end{document}
